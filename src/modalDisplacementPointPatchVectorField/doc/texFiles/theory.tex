\section{Theoretical formulation} \label{sec:theory}

We consider structures composed of an isotropic and homegeneous
elastic material. We assume that the deformations are sufficiently
small so that linear elasticity is a good approximation.
Furthermore, we typicall consider constant density, $\rho_s$,
and elasticity parameters.

We want to represent the motion of the structure by a relatively
small number of modes.
For this purpose, we perform a modal analysis of the structure, as a
pre-processing step before the FSI-problem is set up.
The output of the modal analysis is a set of $N$ modes and $N$ frequencies,
\[
\bphi_i(\bx),\,\,\omega_i,\hspace{1cm} i=1,\ldots,N.
\]
We use bold letters for vector quantities and normal font for scalar
quantities. The modes should be normalized using the density
of the solid in the following way.

\begin{equation}
\int_{V_s} \rho_s(\bx) |\bphi(\bx)|^2\,\mbox{d}V=1
\label{eq:volint}
\end{equation}

Here $V_S$ is the volume occupied by the structure (in the undeformed state),
and we note that the density typically is constant and can be ``moved
outside'' the integral.

Using the modes, we can represent the displacement, $\bu$, of the structure as,
\[
\bu(\bx,t)=\sum_{i=1}^N\alpha_i(t)\bphi_i(\bx),
\]
where we have introduced the time dependent mode coefficients, $\alpha_i$.

The fluid-structure coupling is obtained through the fluid forces on
the surface of the structure. In the initial stage we only include the
pressure forces, but at a later stage it may be useful to also include
the viscous forces. Thus each structural coefficient satisfies an ODE,
with a forcing term containing fluid forces.
\[
\frac{\mbox{d}^2\alpha_i}{\mbox{d}t^2}+
\omega_i^2\alpha_i=Q_i
\]
The forcing is given by the following expression.
\[
Q_i=-\int_S p\bn\cdot \bphi_i\,\mbox{d}S
\]
Here, $\bn$, is the local  unit normal of the surface pointing into the fluid.
